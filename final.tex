% Created 2021-05-10 Mon 17:33
% Intended LaTeX compiler: pdflatex
\RequirePackage{rotating}
\documentclass[12pt]{amsart}
\usepackage[utf8]{inputenc}
\usepackage{lmodern}
\usepackage{graphicx}
\usepackage{longtable}
\usepackage{float}
\usepackage{wrapfig}
\usepackage{rotating}
\usepackage[normalem]{ulem}
\usepackage{textcomp}
\usepackage{marvosym}
\usepackage{wasysym}
\usepackage{hyperref}
\tolerance=1000
\usepackage[authordate-trad,backend=biber,natbib]{biblatex-chicago}
\addbibresource{main.bib}
\addbibresource{ligon.bib}
\renewcommand{\refname}{}
\usepackage{booktabs}
\usepackage{minted}
\newcommand{\T}{\top}
\newcommand{\E}{\ensuremath{\mbox{E}}}
\usepackage{fullpage}
\renewcommand{\thesection}{\Roman{section}}
\author{Ethan Ligon}
\date{\today}
\title{ARE212 Final Exam}
\hypersetup{
 pdfauthor={Ethan Ligon},
 pdftitle={ARE212 Final Exam},
 pdfkeywords={},
 pdfsubject={},
 pdfcreator={Emacs 28.0.50 (Org mode 9.3.7)}, 
 pdflang={English}}
\begin{document}

\maketitle

This is the final exam for ARE212, covering material from the second
half of the course taught in Spring 2021.   The exam is
``take-at-home''; you may consult any resources you wish in completing
it (notes, textbooks, lecture videos, etc.) except for other
people.  This last restriction isn't easily enforceable; I rely on
you to approach this as principled adults who adhere to the
Berkeley Honor Code.

More guidance:
\begin{itemize}
\item The exam is due at 11am  on Tuesday May 11.
\item In completing the exam you should develop written arguments
(e.g., expressed using \LaTeX{} or pencil and paper).  In some cases
you may wish to supplement these written arguments with
computation, such as Monte Carlo experiments.  Should you do so,
please provide me with your working, open source, well-documented code.  (This
last could be links to a github repo, a Jupyter notebook attached
to an email, or similar).  In any case please be sure that
materials you submit are well-organized and clearly
documented---if I overlook some file you've sent or can't run it
that's on you.
\item You are welcome (and even encouraged) to use arguments developed in our \texttt{piazza}
discussions, but in this case please clearly cite the person and
discussion (e.g., ``As argued by Aaron in a discussion `Tests of
Normality' (\texttt{@32\_f3})
the optimal weighting matrix can be written as a function of a
single unknown parameter.'')
\item Please email files or links to \texttt{ligon@berkeley.edu}.
\item If you have questions about the final I will look for these on the
\#are212-econometrics channel of the \href{https://app.slack.com/client/T01A34D5B6V/C01KC6ZBG4C}{FY slack instance}, but I do
not intend to be continuously available on-line, so much better if
you can ask questions early!
\end{itemize}

\section{Identifying assumptions for regression}
\label{sec:org7b183fe}
For each of the questions in this section provide a short answer and
argument.  Note the quality and concision of the argument 
matters much more than the answer!
\begin{enumerate}
\item Evaluate the truth of following statement: ``In the linear
regression \(y=X\beta + u\) the usual identifying assumption
\(\E(u|X)=0\) implies \(\E(h(X)\cdot u)=0\) for any function \(h\)
satisfying some regularity conditions related to measurability.''
\item Consider the same linear regression \(y=X\beta + u\), but now
suppose an alternative identifying assumption \(\E(X|u)=0\).  
Construct a simple estimator based on this alternative.  Compare
the usual and alternative identifying assumptions; are they
equivalent?  Is one stronger than the other?
\item Suppose that \(y=f(X)+u\) for some unknown but continuous function
\(f\).  Suppose we want to use observed data on \(X\) to predict
outcomes \(y\), and seek a predictor \(\hat{y}(X)\) which is ``best''
in the sense that the mean squared prediction error \(\E
     (y-\hat{y}(X)|X)^2\) is minimized.  What can we say about
\(\hat{y}\) and its relation to the conditional expectation
\(\E(y|X)\)?  Its relation to \(u\)?
\end{enumerate}

\section{Omitted Variables}
\label{sec:orgca4950d}
You are asked to serve as a referee for a paper submitted to a top
field journal.  In the submitted paper the researcher uses a sample
of size \(N\) to estimate a model
\[
     y = \alpha + \beta x + u.
  \]
The coefficient \(\beta\) seems to be significantly different from
zero, but the researcher is concerned about omitted variable bias,
so they also estimate a variety of alternative specifications of the form
\[
     y = \alpha + \beta x + \gamma w + u,
  \]
where \(w\) is one of a number of other variables that the researcher
hypothesizes might have some effect on \(y\) as a way of testing the
first model.

The researcher finds a particular variable \(w\) which enters the
regression significantly, and so (i) rejects the first model,
concluding that the first estimate of \(\beta\) was in fact affected by
omitted variable bias; (ii) declares the augmented regression to be
their ``preferred specification;'' and (iii) proceeds to construct
standard \(t\)-statistics for \(\beta\) and \(\gamma\) as a way of
proceeding with inference.

Peer reviews in economics usually include some ``notes for the
author.''  What might your notes say about the paper's approach to
omitted variable bias?  Comment specifically on each of (i), (ii),
and (iii).  Try to make your remarks critical yet
constructive---what shortcomings do you see, and how might the
author address these?

\section{Breusch-Pagan Extended}
\label{sec:org052bc99}
Consider a linear regression of the form
\[ 
     y = \alpha + \beta x + u,
  \]
with \((y,x)\) both scalar random variables, where it is assumed that
(a.i) \(\E(u\cdot x) = \E u = 0\) and (a.ii) \(\E(u^2|x)=\sigma^2\).  
\begin{enumerate}
\item The condition a.i is essentially untestable, but
\cite{breusch-pagan79} argue that one can test a.ii via an
auxiliary regression \(\hat{u}^2 = c + d x + e\), where the \(\hat{u}\)
are the residuals from the first regression, and the test of a.ii
then becomes a test of \(H_0:d=0\).  Explain both why a.i is untestable, and the logic of
the test of a.ii.
\item Use the two conditions a.i and a.ii to construct a GMM version of
the Breusch-Pagan test.
\item What can you say  about the performance or relative merits of the
Bruesch-Pagan test versus your GMM alternative?
\item Suppose that in fact that \(x\) is distributed uniformly over the
interval \([0,2\pi]\), and \(\E(u^2|x)=\sigma^2(x)=\sigma^2+\sin(x)\)
(with \(\sigma^2\) a constant greater than or equal to one), thus
violating a.ii.  What can you say about the performance of the
Breusch-Pagan test in this circumstance?  Can you modify your GMM
test to provide a superior alternative?
\item In the above, we've considered a test of a specific functional
form for the variance of \(u\).  Suppose instead that we don't have
any prior information regarding the form of \(\E(u^2|x)=f(x)\).
Discuss how you might go about constructing an extended version
of the Breusch-Pagan test which tests for \(f(x)\) non-constant.
\item Show that you can use your ideas about estimating \(f(x)\) to
construct a more efficient estimator of \(\beta\) if \(f(x)\) isn't
constant.  Relate your estimator to the optimal generalized least
squares (GLS) estimator.
\end{enumerate}

\section{Black Lives Matter}
\label{sec:orgb50b4b9}
\cite{fryer19} uses data on encounters between police and civilians
of different races in the US to explore how police use of force is
related to a civilian's race.  While Fryer finds that Black and
Hispanic civilians are much more likely to ``experience some form of
force'' from the police and while the probability of being shot by the police is
much higher for a civilian who is Black or Hispanic, Fryer's
most prominent result is that for ``the most extreme use of
force---officer-involved shootings---we find no racial differences
either in the raw data or when contextual factors are taken into
account.''  

Introducing some notation, let \(R\) denote the civilian's race; \(U\)
some variables observed by the police officer prior to any
interaction (e.g., observing ``suspicious'' behavior) but not the
econometrician; \(D\) a binary variable indicating the event (\(D=1\)) of an
encounter between a given civilian and a police officer; \(V\) a set
of ``contextual factors'' related to the encounter and reported by the
officer; and \(S\) the event that the civilian is shot by the officer.
We can then express Fry's finding regarding shootings as not being
able to reject either
\begin{equation}
\label{eq:fry1}
    \mbox{Pr}(S|D=1,R) = \mbox{Pr}(S|D=1)
\end{equation}
or
\begin{equation}
\label{eq:fry2}
    \mbox{Pr}(S|D=1,V,R) = \mbox{Pr}(S|D=1,V).
\end{equation}

\begin{enumerate}
\item \cite{durlauf-heckman20} criticize this conclusion of Fryer's, on
the grounds that \(D\) may be an endogenous variable.  You needn't
read their paper, but explain in your own words what sorts of
endogeneity might undermine Fryer's conclusion that the
probability of being shot by the police doesn't depend on race.

\item Spell out conditions on \((R,S,U,V,D)\) (perhaps using a causal diagram)
which would suffice to interpret (1) and (2) as evidence
that there are no racial differences in the victims of police
shootings.  In particular, what does one need to assume about
\(\mbox{Pr}(D=1|R,U)\)?

\item Consider an alternative (``driving while Black'') model in which
the police use race as a criterion for stopping or otherwise
interacting with a given civilian.  Compare the causal structure
of this model with your answer to (2).  Viewed through the lens
of this model, how would one interpret Fry's failure to reject
\(\mbox{Pr}(S|D=1,R) = \mbox{Pr}(S|D=1)\)?

\item The Justice Department should care\footnote{The Equal Protection
Clause of the fourteenth amendment to the US constitution is
generally interpreted to require ``equality before the law'' for
all persons subject to the jurisdiction of the various states,
and the adoption of this amendment shortly after the Civil War
is regarded as evidence that it was specifically intended to
prevent discrimination against Black Americans.} about which
(Fry's or the ``driving while Black'') is the better model.  How
might one go about testing one against the other?
\end{enumerate}

\section{Weighting of Linear IV Estimators}
\label{sec:orgeb9d69d}
Consider the Linear IV model
\[
      y = X\beta + u\qquad \E(Z^\T u)=0.
  \]
\begin{enumerate}
\item Exploiting the moment condition, under what conditions does the
estimator \(b_{IV}\) satisfying \(Z^\T y = (Z^\T X)b_{IV}\)
consistently estimate \(\beta\)?
\item Suppose that \(Z\) has \(\ell\) columns.  Suppose that \(W\) is a symmetric,
\(\ell\times\ell\) full rank matrix, with a corresponding estimator \(b_W\)
satisfying \(WZ^\T y = W(Z^\T X)b_{W}\).  Under what conditions
will this estimator consistently estimate \(\beta\)?
\item Describe the GMM criterion function that \(b_W\) minimizes.
\item Consider Hansen's description of the two-stage least squares
estimator (Section 12.12).  What is \(W\) for this estimator?
Under what conditions is this the optimally weighted GMM estimator?
\item \(W=I\) for the \(b_{IV}\) estimator described above.  Under what
conditions is \(b_{IV}\) the optimally weighted GMM estimator?
\item For the estimator described in (2), suppose that \(W\) is diagonal,
with diagonal elements proportional to
\((1,1/2,1/4,...,2^{1-\ell})\).  Under what conditions is the
estimator \(b_W\) the optimally weighted estimator?  Can you think
of a practical example where the optimal weighting matrix might
have this structure?
\end{enumerate}


\printbibliography
\end{document}