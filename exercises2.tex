% Created 2022-03-28 Mon 09:39
% Intended LaTeX compiler: pdflatex
\RequirePackage{rotating}
\documentclass[12pt]{amsart}
\usepackage[utf8]{inputenc}
\usepackage{lmodern}
\usepackage{graphicx}
\usepackage{longtable}
\usepackage{float}
\usepackage{wrapfig}
\usepackage{rotating}
\usepackage[normalem]{ulem}
\usepackage{textcomp}
\usepackage{marvosym}
\usepackage{wasysym}
\usepackage{hyperref}
\usepackage{minted}
\usepackage{wasysym}
\newcommand{\Cov}{\ensuremath{\mbox{Cov}}}
\renewcommand{\Pr}{\ensuremath{\mbox{Pr}}}
\newcommand{\Eq}[1]{(\ref{eq:#1})}
\usepackage{bm}\usepackage{econometrics}
\newcommand{\T}{\top}
\newtheorem{proposition}{Proposition} \newcommand{\Prop}[1]{Proposition \ref{prop:#1}}
%\newtheorem{problem}{Problem} \newcommand{\Prob}[1]{Problem \ref{prob:#1}}
%\newtheorem{theorem}{Theorem} \newcommand{\Thm}[1]{Theorem \ref{thm:#1}}
%\newtheorem{corollary}{Corollary} \newcommand{\Cor}[1]{Corollary \ref{cor:#1}}
%\newtheorem{remark}{Remark} \newcommand{\Rem}[1]{Remark \ref{rem:#1}}
%\newtheorem{condition}{Condition} \newcommand{\Cond}[1]{Condition \ref{cond:#1}}
%\newtheorem{lemma}{Lemma} \newcommand{\Lem}[1]{Lemma \ref{lem:#1}}
%\newtheorem{assumption}{Assumption} \newcommand{\Ass}[1]{Assumption \ref{ass:#1}}
\newcommand{\Fig}[1]{Figure \ref{fig:#1}} \newcommand{\Tab}[1]{Table \ref{tab:#1}}
\usepackage{dsfont}\newcommand{\one}{\ensuremath{\mathds{1}}}
\usepackage{xcolor}
\newcommand{\rv}[1]{\ensuremath{\textcolor{red}{#1}{}}}
%\newcommand{\rv}[1]{\ensuremath{{}_{rv}{#1}{}}}
%\newcommand{\rv}[1]{\ensuremath{\underline{#1}{}}}
\newcommand{\rvy}{\rv{y}}
\newcommand{\rvX}{\rv{X}}
\newcommand{\rvx}{\rv{x}}
\newcommand{\rvu}{\rv{u}}
\renewcommand{\do}[1]{\ensuremath{\mbox{do}(#1)}}
\renewcommand{\E}{\ensuremath{\mathds{E}}}
\DeclareMathOperator*{\argmax}{arg\,max}
\DeclareMathOperator*{\argmin}{arg\,min}
\author{Ethan Ligon}
\date{Due April 3, 2022}
\title{Exercises (Non-parametric estimation)}
\hypersetup{
 pdfauthor={Ethan Ligon},
 pdftitle={Exercises (Non-parametric estimation)},
 pdfkeywords={},
 pdfsubject={},
 pdfcreator={Emacs 28.0.50 (Org mode 9.5.2)}, 
 pdflang={English}}
\begin{document}

\maketitle
\begin{enumerate}
\item Let \(f(x)\) be a density, and let \(\hat{f}(x)=\frac{1}{nh}\sum_i
     k\left(\frac{\rvX_i-x}{h}\right)\) be an estimator of \(f(x)\).
These aren't generally equal, of course.  But show that if \(k\) is
a valid positive kernel then \(\hat{f}\) is also a density. (Hint:
use fact that \(k\) is a density and use rules for transforming
random variables.)
\item The density for the standard Cauchy distribution is
\(f(x)=\frac{1}{\pi(1+x^2)}\).  Apparently this could be used as a
kernel.  Of the desirable properties we've discussed
(non-negative, boundedness, symmetry, differentiability) which
are possessed by the Cauchy kernel?
\item In lecture we derived an expression for the bias of \(\hat{f}(x)\)
in terms of the true density \(f\).  Use an analogous argument to
obtain an expression for the variance in terms of \(f\).
\item You use data from the Indian NSS to produce a figure describing
the distribution of non-durable expenditures across households,
measured in INR, and using a kernel estimator with bandwidth \(h\).
A referee asks you to re-estimate the distribution after
converting the expenditure data so that the units are in
USD, using the 2014 PPP rate of 18.4 INR/USD.  What new bandwidth
should you use?
\item There's a sense in which the Epanechnikov kernel, 
\[
     {}  k(u) = \begin{cases} 
                  \frac{3}{4\sqrt{5}}\left(1-\frac{u^2}{5}\right) &
     \text{if $|u|<\sqrt{5}$}\\
     {}              0 & \text{otherwise} \end{cases}
     \] 
is optimal (it minimizes the asymptotic integrated mean square
error of the estimator).  Using the same code we developed in
lecture as a base, implement the Epanechnikov kernel, and compare
its performance in terms of the Oracle calculations of bias and
variance with the Gaussian kernel.
\item Consider the non-parametric regression model
   \begin{align*}
 \rvy &= m(\rvX) + \rvu\\
 \E(\rvu|\rvX) &= 0\\
 \E(\rvu^2|\rvX) &= \sigma^2(\rvX).
\end{align*}
Suppose that (unknown to the econometrician) the `true' \(m(x) =
     \alpha + \beta x\).  The econometrician estimates the
 non-parametric model using kernel regression.  What is the bias
 of the estimator?  How does this depend on the sign of \(\beta\)?
\end{enumerate}
\end{document}